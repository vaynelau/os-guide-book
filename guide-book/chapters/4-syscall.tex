\chapter{系统调用与fork}

\section{实验目的}
  \begin{enumerate}
    \item 掌握系统调用的概念及流程
    \item 实现进程间通讯机制
    \item 实现fork函数
    \item 掌握缺页中断的处理流程
  \end{enumerate}

一般情况下,进程不能够存取系统内核的地址空间,也就是说它不能存取内核使用的内存数据,也不能调用内核函数,
这一点是由CPU的硬件结构保证的。然而,用户进程在特定的场景下是需要进行一些只能在内核中执行的操作,如
对硬件的操作。这种时候允许内核执行用户提供的代码显然是不安全的,所以操作系统也就设计了一系列内核空间
的函数,当用户进程以特定的方式陷入异常后,能够由内核调用对应的函数,我们把这些函数称为\textbf{系统调用}。
在这一节的实验中,我们需要实现系统调用机制,并在此基础上实现进程间通信(IPC)机制和一个重要的系统调用fork。
在fork的实验中,我们会介绍一种被称为写时复制的特性,而与这种特性相关的正是内核的缺页中断处理机制。

\section{系统调用(System Call)}
本节中,我们着重讨论系统调用的作用,并完成实现相关的内容。

\subsection{一探到底,系统调用的来龙去脉}
说起系统调用,你冒出的第一个问题一定是:系统调用到底长什么样子?为了一探究竟,我们选择一个极为简单的程序作为实验对象。
在这个程序中,我们通过puts来输出一个字符串到终端。

\begin{minted}[linenos]{c}
#include <stdio.h>

int main() {
        puts("Hello World!\n");
        return 0;
}
\end{minted}

\begin{note}
如果你还记得C语言课上关于标准输出的相关知识的话,你一定知道在C语言中,终端被抽象为了标准输出文件stdout。
通过向标准输出文件写东西,就可以输出内容到屏幕。而向文件写入内容是通过write系统调用完成的。
因此,我们选择通过观察puts函数,来探究系统调用的奥秘。
\end{note}

我们通过GDB进行单步调试,逐步深入到函数之中,观察puts具体的调用过程\footnote{这里为了方便大家在自己的机器上重现,
我们选用了Linux X86\_64平台作为实验平台}。运行GDB,将断点设置在puts这条语句上,并通过stepi指令\footnote{为了加快调试进程,
可以尝试stepi N指令,N的位置填写任意数字均可。这样每次会执行N条机器指令。笔者使用的是stepi 10。}
单步进入到函数中。当程序到达write函数时停下,因为write正是Linux的一条系统调用。我们打印出此时的函数调用栈,
可以看出,C标准库中的puts函数实际上通过了很多层函数调用,最终调用到了底层的write函数进行真正的屏幕打印操作。

\begin{minted}[linenos]{objdump}
(gdb)
0x00007ffff7b1b4e0 in write () from /lib64/libc.so.6
(gdb) backtrace
#0  0x00007ffff7b1b4e0 in write () from /lib64/libc.so.6
#1  0x00007ffff7ab340f in _IO_file_write () from /lib64/libc.so.6
#2  0x00007ffff7ab2aa3 in ?? () from /lib64/libc.so.6
#3  0x00007ffff7ab4299 in _IO_do_write () from /lib64/libc.so.6
#4  0x00007ffff7ab462b in _IO_file_overflow () from /lib64/libc.so.6
#5  0x00007ffff7ab5361 in _IO_default_xsputn () from /lib64/libc.so.6
#6  0x00007ffff7ab3992 in _IO_file_xsputn () from /lib64/libc.so.6
#7  0x00007ffff7aaa4ef in puts () from /lib64/libc.so.6
#8  0x0000000000400564 in main () at test.c:4
\end{minted}

通过gdb显示的信息,我们可以看到,这个write()函数是在libc.so这个动态链接库中的,也就是说,它仍然是C库中的函数,
而不是内核中的函数。因此,该write()函数依旧是个用户空间函数。为了彻底揭开这个函数的秘密,我们对其进行反汇编。

\begin{minted}[linenos]{objdump}
(gdb) disassemble 0x00007ffff7b1b4e0
Dump of assembler code for function write:
=> 0x00007ffff7b1b4e0 <+0>:     cmpl   $0x0,0x2bf759(%rip)        # 0x7ffff7ddac40
   0x00007ffff7b1b4e7 <+7>:     jne    0x7ffff7b1b4f9 <write+25>
   0x00007ffff7b1b4e9 <+9>:     mov    $0x1,%eax
   0x00007ffff7b1b4ee <+14>:    syscall
   0x00007ffff7b1b4f0 <+16>:    cmp    \$0xfffffffffffff001,%rax
   0x00007ffff7b1b4f6 <+22>:    jae    0x7ffff7b1b529 <write+73>
   0x00007ffff7b1b4f8 <+24>:    retq
End of assembler dump.
\end{minted}

通过gdb的反汇编功能,我们可以看到,这个函数最终执行了syscall这个极为特殊的指令。从它的名字我们就能够猜出它的用途,
它使得进程陷入到内核态中,执行内核中的相应函数,完成相应的功能。在系统调用返回后,用户空间的相关函数会将系统调用的结果,
通过一系列的过程,最终返回给用户程序。

由此我们可以看到,系统调用实际上是操作系统和用户空间的一组接口。用户空间的程序通过系统调用来访问操作系统的一些服务,
谋求操作系统提供必要的帮助。

在进行了上面的一系列探究后,我们将我们的发现罗列出来,整理一下我们的思路:
\begin{itemize}
  \item 存在一些只能由操作系统来完成的操作(如读写设备、创建进程等)。
  \item 用户程序要实现一些功能(比如执行另一个程序、读写文件),必须依赖操作系统的帮助。
  \item C标准库中的一些函数的实现必须依赖于操作系统(如我们所探究的puts函数)。
  \item 通过执行syscall指令,我们可以陷入到内核态,请求操作系统的一些服务。
  \item 直接使用操作系统的功能是很繁复的(每次都需要设置必要的寄存器,并执行syscall指令)
\end{itemize}

之后,我们再来整理一下调用C标准库中的puts函数的过程中发生了什么:
\begin{enumerate}
  \item 调用puts函数
  \item 在一系列的函数调用后,最终调用了write函数。
  \item write函数为寄存器设置了相应的值,并执行了syscall指令。
  \item 进入内核态,内核中相应的函数或服务被执行。
  \item 回到用户态的write函数中,将系统调用的结果从相关的寄存器中取回,并返回。
  \item 再次经过一系列的返回过程后,回到了puts函数中。
  \item puts函数返回。
\end{enumerate}

综合上面这些内容,相信你一定已经发现了其中的巧妙之处。操作系统将自己所能够提供的服务以系统调用的方式提供给用户空间。
用户程序即可通过操作系统来完成一些特殊的操作。同时,所有的特殊操作就全部在操作系统的掌控之中了
(因为用户程序只能通过由操作系统提供的系统调用来完成这些操作,所以操作系统可以确保用户不破坏系统的稳定)。
而直接使用这些系统调用较为麻烦,于是由产生了用户空间的一系列API,如POSIX、C标准库等,它们在系统调用的基础上,
实现更多更高级的常用功能,使得用户在编写程序时不用再处理这些繁琐而复杂的底层操作,
而是直接通过调用高层次的API就能实现各种功能。通过这样巧妙的层次划分,使得程序更为灵活,也具有了更好的可移植性。
对于用户程序来说,只要自己所依赖的API不变,无论底层的系统调用如何变化,都不会对自己造成影响,
使得程序更易于在不同的系统间移植。整个结构如表\ref{fig:api-and-syscall}所示。

\begin{table}[htbp]
\centering
\caption{API、系统调用层次结构}
\label{fig:api-and-syscall}
\begin{tabular}{|lr|}
\hline
\multicolumn{2}{|c|}{用户程序 User Program}        \\ \hline
应用程序编程接口 API & POSIX, C Standard Library, etc. \\ \hline
系统调用         & read, write, fork, etc.         \\ \hline
\end{tabular}
\end{table}

\newpage

\subsection{系统调用机制的实现}
在发现了系统调用的本质之后,我们就要着手在我们的小操作系统中实现一套系统调用机制了。不过不要着急,为了使得后面的思路更清晰,
我们先来整理一下系统调用的流程:
\begin{enumerate}
  \item 调用一个封装好的用户空间的库函数(如writef)
  \item 调用用户空间的syscall\_*函数
  \item 调用msyscall,用于陷入内核态
  \item 陷入内核,内核取得信息,执行对应的内核空间的系统调用函数(sys\_*)
  \item 执行系统调用,并返回用户态,同时将返回值“传递”回用户态
  \item 从库函数返回,回到用户程序调用处
\end{enumerate}

在用户空间的程序中,我们定义了许多的函数,以writef函数为例,这一函数实际上并不是最接近内核的函数,它最后会调用一个名为syscall\_putchar
的函数,这个函数在user/syscall\_lib.c中。在我们的小操作系统实验中,这些syscall开头的函数与内核中的系统调用函数(sys开头的函数)是一一对应的,
syscall开头的函数是我们在用户空间中最接近的内核的也是最原子的函数,而sys开头的函数是内核中系统调用具体内容。
syscall开头的函数的实现中,它们毫无例外都调用了msyscall函数,而且函数的第一个参数都是一个与调用名相似的宏(如SYS\_putchar),
在我们的小操作系统实验中把这个参数称为\textbf{系统调用号}(请找到这个宏的定义,了解系统调用号的排布规则),系统调用号是
内核区分这究竟是何种系统调用的唯一依据。除此之外msyscall函数还有5个参数,这些参数是系统调用实际需要使用的参数,而为了方便我们使用了
取了最多参数的系统调用所需要的参数数量(syscall\_mem\_map函数具有5个参数)。

在syscall\_*系列函数中,我们将参数传递给了msyscall函数,而这些参数究竟是如何安置的呢?这里就需要用MIPS的调用规范来说明这件事情了,
我们把函数体中没有函数调用语句的函数称为\textbf{叶函数},自然如果有函数调用语句的函数称为非叶函数。在MIPS的调用规范中,进入函数体时会通过对栈指针做减法的方式
为自身的局部变量、返回地址、调用函数的参数分配存储空间(叶函数没有后两者),在函数调用结束之后会对栈指针做加法来释放这部分空间,我们把
这部分空间称为\textbf{栈帧(Stack Frame)}。非叶函数是在调用方的栈帧的底部预留被调用函数的参数存储空间(被调用方从调用方函数的栈帧中取得参数)。
以我们的小操作系统为例,msyscall函数一共有6个参数,前4个参数会被syscall开头的函数分别存入\$a0-\$a3寄存器(寄存器传参的部分)同时栈帧底部保留16字节的空间(不要求存入参数的值),
后2个参数只会被存入在前4的参数的预留空间之上的8字节空间内(没有寄存器传参)。这些过程虽然不需要我们显式地编写汇编来完成,但是需要在内核中是以汇编的方式显式地把函数的参数值“转移”到内核空间中的。

既然参数的位置已经被合理安置,那么接下来我们需要编写msyscall函数,这个叶函数没有局部变量,也就是说这个函数不需要分配栈帧,我们\textbf{只}需要执行
特权指令(syscall)来陷入内核态以及函数调用返回即可。

\begin{exercise}
填写user/syscall\_wrap.S中的msyscall函数,使得用户部分的系统调用机制可以正常工作。
\end{exercise}

在通过特权指令syscall陷入内核态后,处理器将PC寄存器指向一个相同的内核异常入口。在
trap\_init函数中将系统调用类型的异常的入口设置为了handle\_sys函数,这一函数在lib/syscall.S中。需要注意的是,此处的栈指针是内核空间的栈指针,内核将运行现场
保存到内核空间后(其保存的结构与结构体\mintinline{c}|struct Trapframe|等同),栈指针指向这个结构体的起始位置,你可以借助include/trap.h的宏使用lw指令取得保存现场的一些寄存器的值。


\begin{codeBoxWithCaption}{内核的系统调用处理程序\label{code:handlesys.S}}
  \inputminted[linenos]{gas}{codes/handlesys.S}
\end{codeBoxWithCaption}

\begin{thinking}\label{think-syscall}
 思考并回答下面的问题:
  \begin{itemize}
   \item 内核在保存现场的时候是如何避免破坏通用寄存器的?
   \item 系统陷入内核调用后可以直接从当时的\$a0-\$a3参数寄存器中得到用户调用msyscall留下的信息吗?
   \item 我们是怎么做到让sys开头的函数“认为”我们提供了和用户调用msyscall时同样的参数的?
   \item 内核处理系统调用的过程对Trapframe做了哪些更改?这种修改对应的用户态的变化是?
  \end{itemize}
\end{thinking}


\begin{exercise}
按照lib/syscall.S中的提示,完成handle\_sys函数,使得内核部分的系统调用机制可以正常工作。
\end{exercise}

做完这一步,整个系统调用的机制已经可以正常工作,接下来我们要来实现几个具体的系统调用。

\subsection{基础系统调用函数}

在系统调用机制搞定之后,我们自然是要弄几个系统调用玩一玩了。我们实现些什么系统调用呢?打开 lib/syscall\_all.c,可以看到玲琅满目的系统调用函数等着我们去填写。这些系统调用都是基础的系统调用,不论是之后的IPC还是fork,都需要这些基础的系统调用作为支撑。
首先我们看向sys\_mem\_alloc。这个函数的主要功能是分配内存,简单的说,用户程序可以通过这个系统调用给该程序所允许的虚拟内存空间内存\textbf{显式地}分配实际的物理内存,需要用到一些我们之前在pmap.c中所定义的函数
\begin{exercise}
实现lib/syscall\_all.c中的int sys\_mem\_alloc(int sysno,u\_int envid, u\_int va, u\_int perm)函数
\end{exercise}
我们再来看sys\_mem\_map,这个函数的参数很多,但是意义也很直接:将源进程地址空间中的相应内存映射到目标进程的相应地址空间的相应虚拟内存中去。换句话说,此时两者共享着一页物理内存。
\begin{exercise}
实现lib/syscall\_all.c中的int sys\_mem\_map(int sysno,u\_int srcid, u\_int srcva, u\_int dstid, u\_dstva, u\_int perm)函数
\end{exercise}
关于内存的还有一个函数:sys\_mem\_unmap, 正如字面意义所显示的,这个系统调用的功能是解除某个进程地址空间虚拟内存和物理内存之间的映射关系。
\begin{exercise}
实现lib/syscall\_all.c中的int sys\_mem\_unmap(int sysno,u\_int envid, u\_int va)函数
\end{exercise}
除了与内存相关的函数外,另外一个常用的系统调用函数是sys\_yield,这个函数的功能主要就在于实现用户进程对CPU的放弃,可以利用我们之前已经编写好的函数,
另外为了通过我们之前编写的进程切换机制保存现场,这里需要在KERNEL\_SP和TIMESTACK上做一点准备工作
\begin{exercise}
实现lib/syscall\_all.c中的void sys\_yield(void)函数
\end{exercise}
可能你也注意到了,在此我们的系统调用函数并没使用到它的第一个参数sysno,在这里,sysno作为系统调用号被传入,现在起的更多是一个”占位“的作用,能和之前用户层面的系统调用函数参数顺序相匹配。
\section{进程间通信机制(IPC)}
在系统调用机制搞定之后,我们自然是要弄几个系统调用玩一玩了。作为一个微内核系统,我们要来实现个什么系统调用呢?
没错,当然是IPC了。IPC可是微内核最重要的机制之一了。

\begin{note}
上世纪末,微内核设计逐渐成为了一个热点。微内核设计主张将传统操作系统中的设备驱动、文件系统等可在用户空间实现的功能,
移出内核,作为普通的用户程序来实现。这样,即使它们崩溃,也不会影响到整个系统的稳定。其他应用程序通过进程间通讯来请求
文件系统等相关服务。因此,在微内核中IPC是一个十分重要的机制。
\end{note}

接下来进入正题,IPC机制远远没有我们想象得那样神秘,特别是在我们这个被极度简化了的小操作系统中。
根据之前的讨论,我们能够得知这样几个细节:

\begin{itemize}
  \item IPC的目的是使两个进程之间可以通讯
  \item IPC需要通过系统调用来实现
\end{itemize}

所谓通信,最直观的一种理解就是交换数据。假如我们能够将让一个进程有能力将数据传递给另一个进程,
那么进程之间自然具有了相互通讯的能力。那么,要实现交换数据,我们所面临的最大的问题是什么呢?
没错,问题就在于\textbf{各个进程的地址空间是相互独立的}。相信你在实现内存管理的时候已经深刻体会到了这一点,
每个进程都有各自的地址空间,这些地址空间之间是相互独立的。因此,要想传递数据,
我们就需要想办法\textbf{把一个地址空间中的东西传给另一个地址空间}。

想要让两个完全独立的地址空间之间发生联系,最好的方式是什么呢?对,我们要去找一找它们是否存在共享的部分。
虽然地址空间本身独立,但是有些地址也许被映射到了同一物理内存上。如果你之前详细地看过进程的页表建立的部分的话,
想必你已经找到线索了。是的,线索就在env\_setup\_vm()这个函数里面。

\begin{minted}[linenos]{c}
static int
env_setup_vm(struct Env *e)
{
    //略去的无关代码

    for (i = PDX(UTOP); i <= PDX(~0); i++) {
        pgdir[i] = boot_pgdir[i];
    }
    e->env_pgdir = pgdir;
    e->env_cr3   = PADDR(pgdir);

    //略去的无关代码
}
\end{minted}

如果你之前认真思考了这个函数的话会发现,所有的进程都共享了内核所在的2G空间。对于任意的进程,这2G都是一样的。
因此,想要在不同空间之间交换数据,我们就需要借助于内核的空间来实现。那么,我们把要传递的消息放在哪里比较好呢?
恩,发送和接受消息和进程有关,消息都是由一个进程发送给另一个进程的。内核里什么地方和进程最相关呢?啊哈!进程控制块!

\begin{minted}[linenos]{c}
struct Env {
    // Lab 4 IPC
    u_int env_ipc_value;            // data value sent to us
    u_int env_ipc_from;             // envid of the sender
    u_int env_ipc_recving;          // env is blocked receiving
    u_int env_ipc_dstva;        // va at which to map received page
    u_int env_ipc_perm;     // perm of page mapping received
};
\end{minted}

果然,我们看到了我们想要的东西,env\_ipc\_value用于存放需要发给当前进程的数据。
env\_ipc\_dstva则说明了接收到的页需要被映射到哪个虚地址上。知道了这些,我们就不难实现IPC机制了。只需要做做赋值,
填充下对应的域,映射下该映射的页之类的就好了。

\begin{exercise}
实现lib/syscall\_all.c中的void sys\_ipc\_recv(int sysno,u\_int dstva)函数和
int sys\_ipc\_can\_send(int sysno,u\_int envid, u\_int value, u\_int srcva, u\_int perm)函数。
\end{exercise}

sys\_ipc\_recv(int sysno,u\_int dstva)函数首先要将env\_ipc\_recving设置为1,表明该进程准备接受其它进程的消息了。
之后阻塞当前进程,即将当前进程的状态置为不可运行。之后放弃CPU(调用相关函数重新进行调度)。

int sys\_ipc\_can\_send(int sysno,u\_int envid, u\_int value, u\_int srcva, u\_int perm)函数用于发送消息。
如果指定进程为可接收状态,则发送成功,清除接收进程的接收状态,使其可运行,返回0,否则,返会\_E\_IPC\_NOT\_RECV。

值得一提的是,由于在我们的用户程序中,会大量使用srcva为0的调用来表示不需要传递物理页面,因此在编写相关函数时也需要注意此种情况。

讲完IPC后你已经可以参照编写用户进程,利用实现好的IPC系统调用来实现一些有意思的小程序了。

\section{FORK}

在Lab3我们曾提到过,env\_alloc是内核产生一个进程。但如果想让一个进程创建一个进程,
就像是父亲与儿子那样,我们就需要使用到fork了。那么fork究竟是什么呢?

\subsection{初窥fork}
fork,直观意象是叉子的意思。在我们这里更像是分叉的意思,就好像一条河流动着,遇到一个分叉口,分成两条河一样,
fork就是那个分叉口。在操作系统中,在某个进程中调用fork()之后,将会以此为分叉分成两个进程运行。
新的进程在开始运行时有着和旧进程\textbf{绝大部分相同的信息},而且在新的进程中fork依旧有
一个返回值,只是该返回值为0。在旧进程,也就是所谓的父进程中,fork的返回值是子进程的env\_id,是大于0的。
在父子进程中有不同的返回值的特性,可以让我们在使用fork后很好地区分父子进程,从而安排不同的工作。

你可能会想,fork执行完为什么不直接生成一个空白的进程块,生成一个几乎和父进程一模一样的子进程有什么用呢?
换成创建一个空白的进程多简单!按笔者的理解,这是因为:
\begin{itemize}
 \item 与不相干的两个进程相比,父子进程间的通信要方便的多。因为fork虽然没法造成进程间的统治关系\footnote{这是因为进程之间是并发的,在操作系统看来,父子进程之间更像是兄弟关系。},
但是因为在子进程中记录了父进程的一些信息,父进程也可以很方便地对子进程进行一些管理等。
 \item  当然还有一个可能的原因在于安全与稳定,尤其是关于操作权限方面。对这方面有兴趣的同学可以查看链接\footnote{http://www.jbxue.com/shouce/apache2.2/mod/prefork.html}
探索一下。
\end{itemize}

fork之后父子进程就分道扬镳,互相独立了。而和fork“针锋相对”却又经常“纠缠不清”的,
是名为exec系列的系统调用。它会“勾引”子进程抛弃现有的一切,另起炉灶。若在子进程中执行exec,
完成后子进程从父进程那拷贝来的东西就全部消失了。取而代之的是一个全新的进程,就像太乙真人用莲藕
为哪吒重塑了一个肉身一样。
exec系列系统调用我们将会作为一个挑战性任务放在后面来实现,暂时不做过多介绍。

为了让你对fork的认识不只是停留在理论层面,我们下面来做一个小实验,复制到你的linux环境下运行一下吧。
\begin{codeBoxWithCaption}{理解fork\label{code:fork_test.c}}
  \inputminted[linenos]{c}{codes/fork_test.c}
\end{codeBoxWithCaption}

使用\mintinline{c}|gcc fork_test.c|,然后\mintinline{c}| ./a.out| 运行一下,你得到的正常的结果应该如下所示:
\begin{minted}[linenos]{console}
Before fork.
After fork.
After fork.
son.pid:16903 (数字不一定一样)
father.pid:16902
\end{minted}

我们从这段简短的代码里可以获取到很多的信息,比如以下几点:
\begin{itemize}
 \item 在fork之前的代码段只有父进程会执行。
 \item 在fork之后的代码段父子进程都会执行\label{fork与子进程}。
 \item fork在不同的进程中返回值不一样,在父进程中返回值不为0,在子进程中返回值为0。
 \item 父进程和子进程虽然很多信息相同,但他们的env\_id是不同的。
\end{itemize}

从上面的小实验我们也能看出来——子进程实际上就是按父进程的绝大多数信息和状态作为模板而雕琢出来的。
即使是以父进程为模板,父子进程也还是有很多不同的地方,不知细心的你从刚才的小实验中能否看出父子进程有哪些地方是明显不一样的吗?

\begin{thinking}\label{think-father-son}
 思考下面的问题,并对这两个问题谈谈你的理解:
  \begin{itemize}
   \item 子进程完全按照fork()之后父进程的代码执行,说明了什么?
   \item 但是子进程却没有执行fork()之前父进程的代码,又说明了什么?
  \end{itemize}
\end{thinking}


\begin{thinking}\label{think-fork的调用}
 关于fork函数的两个返回值,下面说法正确的是:

  A、fork在父进程中被调用两次,产生两个返回值

  B、fork在两个进程中分别被调用一次,产生两个不同的返回值

  C、fork只在父进程中被调用了一次,在两个进程中各产生一个返回值

  D、fork只在子进程中被调用了一次,在两个进程中各产生一个返回值
\end{thinking}


% \item 表明子进程拷贝了父进程的代码段。
 %\item 子进程却没有执行fork()之前父进程的代码,表明子进程开始运行时的PC值不是二进制镜像的入口!
 %\item fork有两个返回值,不是指fork在同一个进程中返回两次,而是指fork在两个进程中均有返回值,且返回值不同。

%限于笔者自身的理解与表述能力,对fork的表述还是含糊不清的,如果想获得更多的理解,推荐查看链接中的帖子:\url{http://bbs.chinaunix.net/forum.php?mod=viewthread&tid=311067}

\subsection{写时复制机制}

通过使用初步了解fork后,先不着急实现它。俗话说“兵马未动,粮草先行”,我们先来了解一下关于fork的底层细节。
根据维基百科的描述,在fork时,父进程会为子进程分配独立的地址空间。但是分配独立的虚拟空间并不意味
着一定会分配额外的物理内存:父子进程用的是相同的物理空间。子进程的代码段、数据段、堆栈
都是指向父进程的物理空间,也就是说,虽然两者的虚拟空间是不同的,但是他们所对应的物理空间是同一个。

\begin{note}
\small{
Wiki Fork: In Unix systems equipped with virtual memory support (practically all modern variants), the fork operation creates a separate address space
 for the child. The child process has an exact copy of all the memory segments of the parent process, though if copy-on-write semantics
 are implemented,the physical memory need not be actually copied. Instead, virtual memory pages in both processes may refer to the same pages of physical memory
 until one of them writes to such a page: then it is copied. This optimization is important in the common case where fork is used
 in conjunction with exec to execute a new program: typically, the child process performs only a small set of actions before it ceases
 execution of its program in favour of the program to be started, and it requires very few, if any, of its parent's data structures.}
\end{note}

那你可能就有问题了:既然上文提到了父子进程之间是独立的,而现在又说共享物理内存,这不是矛盾吗?
按照共享物理内存的说法, 那岂不是变成了“父教子从,子不得不从”?

这两种说法实际上不矛盾,因为父子进程共享物理内存是有前提条件的:共享的物理内存不会被任一进程修改。那么,对于那些父进程或子进程修改的内存我们又该如何处理呢?
这里我们引入一个新的概念——写时复制(Copy On Write,简称COW)。通俗来讲就是当父子进程中有\textbf{修改}内存(一般是数据段)的行为发生时,内核捕获这种缺页中断后,再为\textbf{发生内存修改的进程}相应的地址分配物理页面,而一般来说子进程的代码段继续共享父进程的物理空间(两者的代码完全相同)。

\begin{note}
如果在fork之后在子进程中执行了exec,由于这时和父进程要执行的代码完全不同,子进程的代码段也会分配单独的物理空间。
\end{note}

在我们的小操作系统实验中,对于所有的可被写入的内存页面,都需要\textbf{通过设置页表项标识位PTE\_COW的方式}被保护起来。
无论父进程还是子进程何时试图写一个被保护的物理页,就会产生一个异常(一般指缺页中断 Page Fault),这一异常的处理会在后文详细介绍。

\begin{note}
早期的Unix系统对于fork采取的策略是:直接把父进程所有的资源复制给新创建的进程。
这种实现过于简单,并且效率非常低。因为它拷贝的内存也许是需要父子进程共享的,
当然更糟的情况是,如果新进程打算通过exec执行一个新的映像,那么所有的拷贝都将前功尽弃。
\end{note}

\subsection{返回值的秘密}

小红:“咦,不科学啊。fork的两个返回值为啥是系统调用syscall\_env\_alloc的功劳?不是说子进程只执行fork之后的代码吗?”

小绿:“你还别不信,还真的就是系统调用syscall\_env\_alloc的功劳。我们前面是提到了子进程执行fork之后的代码,实则不准确:因为在fork内部呀,
就要用sys\_env\_alloc的两个返回值区分开父子进程,好安排他们在返回之后执行不同的任务呀!你想想,虽然子进程在被创建出来就已经有了
进程控制块和进程上下文,但是子进程是否能够开始被调度是要由父进程决定的。

在我们的小操作系统实验中,需要强调的一点是我们实现的fork是一个用户态函数,fork函数中需要若干个“原子的”系统调用来完成所期望的功能。其中最核心的一个系统调用就是一个新的进程的创建syscall\_env\_alloc。

在fork的实现中,我们是通过判断syscall\_env\_alloc的返回值来决定fork的返回值以及后续动作,所以会有类似这样结构的代码片段:
\begin{minted}[linenos]{c}
 envid = syscall_env_alloc();
 if (envid == 0) {
     // 子进程
     ...
 }
 else {
     // 父进程
     ...
 }
\end{minted}

既然fork的目的是使得父子进程处于几乎相同的运行状态,我们可以认为它们都应该经历了同样的“恢复运行现场”的过程,只不过对于父进程是从系统调用中返回的恢复现场,
而对于子进程则是在进程调度时进行的现场恢复。在现场恢复后,进程会从同样的地方返回到fork函数中。而它们携带的函数的返回值是不同的,这也就能够在fork函数中区分两者。

为了实现这一特性,你可能需要先实现sys\_env\_alloc的几个任务,它除了创建一个新的进程外,还需要用一些当前进程的信息作为模版来填充这个进程:

\begin{description}
 \item [运行现场] 要复制一份当前进程的运行现场Trapframe到子进程的进程控制块中。
 \item [程序计数器] 子进程的程序计数器应该被设置为syscall\_env\_alloc返回后的地址,也就是它陷入异常地址的下一行指令的地址,\textbf{这个值已经存在于Trapframe中}。
 \item [返回值有关] 这个系统调用本身是需要一个返回值的(这个返回过程只会影响到父进程),对于子进程则需要对它的运行现场Trapframe进行一个修改。
 \item [进程状态] 我们当然不能让子进程在父进程syscall\_env\_alloc返回后就直接进入调度,因为这时候它还没有做好充分的准备,所以我们需要设定不能
 让它被加入调度队列。
 \end{description}

\begin{exercise}
填写 lib/syscall\_all.c 中的 sys\_env\_alloc 函数
\end{exercise}

在解决完返回值的问题之后,父与子就能够分别走上各自的旅途了。

\subsection{父子各自的旅途}

进程在很多时候(如进程通信)都是需要访问自身的进程控制块的,用户程序初次运行时会将一个\mintinline{c}| struct Env *env|
指针指向自身的进程控制块。作为子进程,它很明显具有了一个与父亲不同的进程控制块,所以在第一次被调度的时候(当然
这时还是在fork函数中)它就需要将env指针指向自身的进程控制块,这个时候还需要通过另一个系统调用来取得自己的
envid,因为对于子进程而言syscall\_env\_alloc返回的是一个0值。做完这一步,子进程就可以从fork函数中返回,开始自己的旅途了。

\begin{exercise}
 填写 user/fork.c 中的fork 函数中关于sys\_env\_alloc的部分和“子进程”执行的部分
\end{exercise}

父亲在儿子醒来之前则需要做更多的准备,而这些准备中最重要的一步是遍历进程的\textbf{大部分用户空间页},
对于所有可以写入的页面的页表项,\textbf{在父进程和子进程都}加以PTE\_COW标志位保护起来。这里需要
实现duppage函数来完成这个过程。

\begin{thinking}\label{think:遍历页}
	如果仔细阅读上述这一段话,你应该可以发现,我们并不是对所有的用户空间页都使用duppage进行了保护。那么究竟哪些用户空间页可以保护,哪些不可以呢,
	请结合 include/mmu.h 里的内存布局图谈谈你的看法。
\end{thinking}

\begin{thinking}\label{think:vpt的使用}
在遍历地址空间存取页表项时你需要使用到vpd和vpt这两个“指针的指针”,请思考并回答这几个问题:
\begin{itemize}
 \item vpt和vpd的作用是什么?怎样使用它们?
 \item 从实现的角度谈一下为什么能够通过这种方式来存取进程自身页表?
 \item 它们是如何体现自映射设计的?
 \item 进程能够通过这种存取的方式来修改自己的页表项吗?
\end{itemize}
\end{thinking}

在duppage函数中,唯一需要强调的一点是要对不同权限的页有着不同的处理方式,你可能会遇到这几种情况:
\begin{description}
 \item [只读页面] 按照相同权限(只读)映射给子进程即可
 \item [共享页面] 即具有PTE\_LIBRARY标记的页面,这类页面需要保持共享的可写的状态
 \item [写时复制页面] 即具有PTE\_COW标记的页面,这类页面是上一次的fork的duppage的结果
 \item [可写页面] 需要给父进程和子进程的页表项都加上PTE\_COW标记
\end{description}

\begin{exercise}
 结合注释,填写 user/fork.c 中的 duppage 函数
\end{exercise}

\begin{note}
在我们的小操作系统实验中实现的fork并不是一整个原子的过程,所以会出现一段时间(也就是在duppage之前的时间)
我们没有来得及为堆栈所在的页面
设置写时复制的保护机制,在这一段时间内对堆栈的修改(比如发生了其他的函数调用),则会将非叶函数syscall\_env\_alloc
函数调用的栈帧中的返回地址覆盖。这一问题对于父进程来说是理所当然的,然而对于子进程来说,这个覆盖导致
的后果则是在从syscall\_env\_alloc返回时跳转到一个不可预知的位置造成panic。当然你现在看到的代码已经通过一个
优雅的办法来修补这个臭虫:与其他系统调用函数不同,syscall\_env\_alloc是一个内联(inline)的函数,也就是说
这个函数并不会被编译为一个函数,而是直接内联展开在fork函数内。所以syscall\_env\_alloc的栈帧就不存在了,而
msyscall函数的返回指令也直接返回到了fork函数内。如此这个困扰了学生若干年的臭虫就解决了。
\end{note}

在完成写时复制的保护机制后,还不能让子进程处于能被调度的状态,因为作为父亲它还有其他的责任——为写时复制特性
的\textbf{缺页中断}处理做好准备。

\subsection{缺页中断}

内核在捕获到
一个常规的\textbf{缺页中断(Page Fault)}时(在MIPS中这个情况特指TLB缺失,因为MIPS不存在MMU只存在TLB,TLB缺失查找填入都是内核以软件编
程的方式完成的),会进入到一个在trap\_init中“注册”的handle\_tlb的内核处理函数中,这一汇编函数的实现在lib/genex.S
中,化名为一个叫do\_refill的函数。如果物理页面在页表中存在,则会将TLB填入并
返回异常地址再次执行内存存取的指令。如果物理页面不存在,则会触发一个一般意义的缺页错误,并跳转到mm/pmap.c中的
pageout函数中,在存取地址合法的情况下,内核会在用户空间的对应地址分配映射一个物理页面(被动的分配页面)来解
决缺页的问题。

前文中我们提到了写时复制特性,而写时复制特性也是依赖于缺页中断的。我们在trap\_init中注册了另外一个处理函数——handle\_mod,这一函数会跳转到lib/traps.c的page\_fault\_handler函数中,这个函数正是
处理写时复制特性的缺页中断的内核处理函数。

你会发现在这个函数中似乎并没有做任何的页面复制操作,这是为什么呢?答案是这样的,在我们的小操作系统实验中按照
微内核的设计理念,会将大部分的功能都从内核移到用户程序,其中也包括了写时复制的缺页中断处理。真正的处理过程
是用户进程自身去完成的。

如果需要用户进程去完成页面复制等处理过程,是不能直接使用原先的堆栈的(因为发生缺页错误
的也可能是正常堆栈的页面),所以这个时候用户进程就需要一个另外的堆栈来执行处理程序,我们把这个堆栈称作
\textbf{异常处理栈},它的栈顶对应的是宏UXSTACKTOP。异常处理栈需要父进程为自身以及子进程分配映射物理页面。
此外内核还需要知晓进程自身的处理函数所在的地址,这个地址存在于进程控制块的env\_pgfault\_handler域中,这个地址
也需要事先由父进程通过系统调用设置。

\begin{exercise}
完成 lib/traps.c 中的 page\_fault\_handler 函数
\end{exercise}

\begin{thinking}\label{think:pgfault-kernel}
page\_fault\_handler 函数中,你可能注意到了一个向异常处理栈复制Trapframe运行现场的过程,请思考并回答这几个问题:
\begin{itemize}
  \item 这里实现了一个支持类似于“中断重入”的机制,而在什么时候会出现这种“中断重入”?
  \item 内核为什么需要将异常的现场Trapframe复制到用户空间?
\end{itemize}
\end{thinking}

让我们回到fork函数,在提示使用syscall\_env\_alloc之前,有另一个提示——使用set\_pgfault\_handler函数来“安装”
处理函数。
\begin{minted}[linenos]{c}
void set_pgfault_handler(void (*fn)(u_int va))
{
  if (__pgfault_handler == 0) {
    if (syscall_mem_alloc(0, UXSTACKTOP - BY2PG, PTE_V | PTE_R) < 0 ||
      syscall_set_pgfault_handler(0, __asm_pgfault_handler, UXSTACKTOP) < 0) {
      writef("cannot set pgfault handler\n");
      return;
    }
  }
  __pgfault_handler = fn;
}
\end{minted}

上面的set\_pgfault\_handler函数中,进程\textbf{为自身}分配映射了异常处理栈,
同时也用系统调用告知内核自身的处理程序是\_\_asm\_pgfault\_handler(在entry.S定义),随后内核也需要就将进程控制块的env\_pgfault\_handler域设为它。在函数的最后,将在entry.S定义的字\_\_pgfault\_handler赋值为fn,而这个fn究竟是什么我们稍后再说。这里需要你完成内核设置进程控制块中的两个域的系统调用。

\begin{exercise}
完成 lib/syscall\_all.c 中的 sys\_set\_pgfault\_handler 函数
\end{exercise}

我们现在知道了缺页中断会返回到entry.S中的\_\_asm\_pgfault\_handler函数,我们再来看这个函数会
做些什么。
\begin{minted}[linenos]{gas}
__asm_pgfault_handler:
lw      a0, TF_BADVADDR(sp)
lw      t1, __pgfault_handler
jalr    t1
nop

lw      v1, TF_LO(sp)
mtlo    v1
lw      v0, TF_HI(sp)
lw      v1, TF_EPC(sp)
mthi    v0
mtc0    v1, CP0_EPC
lw      $31, TF_REG31(sp)

lw      $1, TF_REG1(sp)
lw      k0, TF_EPC(sp)
jr      k0
lw      sp, TF_REG29(sp)
\end{minted}

从内核返回后,此时的栈指针是由内核设置的在异常处理栈的栈指针,而且指向一个由内核复制好的Trapframe结构体的底部。
通过宏TF\_BADVADDR用lw指令取得了Trapframe中的cp0\_badvaddr字段的值,这个值
也正是发生缺页中断的虚拟地址。将这个地址作为第一个参数去调用了\_\_pgfault\_handler这个字内存储的函数,不难看出这个函数是真正进行处理的函数。
函数返回后就是一段类似于恢复现场的汇编,最后非常巧妙地利用了MIPS的延时槽特性跳转的同时恢复了栈指针。

\begin{thinking}\label{think:pgfault-user-1}
到这里我们大概知道了这是一个由用户程序处理并由用户程序自身来恢复运行现场的过程,请思考并回答以下几个问题:
\begin{itemize}
  \item 用户处理相比于在内核处理写时复制的缺页中断有什么优势?
  \item 从通用寄存器的用途角度讨论用户空间下进行现场的恢复是如何做到不破坏通用寄存器的?
\end{itemize}
\end{thinking}

说到这里,我们就要来实现真正进行处理的函数:user/fork.c中的pgfault函数了,pgfault需要完成这些任务:
\begin{enumerate}
	\item 判断页是否为写时复制的页面,是则进行下一步,否则报错
	\item 分配一个新的内存页到临时位置,将要复制的内容拷贝到刚刚分配的页中
	\item 将临时位置上的内容映射到发生缺页中断的虚拟地址上,然后解除临时位置对内存的映射
\end{enumerate}

\begin{exercise}
填写 user/fork.c 中的 pgfault函数
\end{exercise}

这里的pgfault也正是父进程在fork中使用set\_pgfault\_handler函数安装的处理函数。

\begin{thinking}\label{think:pgfault-user-2}
请思考并回答以下几个问题:
\begin{itemize}
  \item 为什么需要将set\_pgfault\_handler的调用放置在syscall\_env\_alloc之前?
  \item 如果放置在写时复制保护机制完成之后会有怎样的效果?
  \item 子进程需不需要对在entry.S定义的字\_\_pgfault\_handler赋值?
\end{itemize}
\end{thinking}

父进程还需要为子进程通过类似于set\_pgfault\_handler函数的方式,用若干系统调用分配子进程的异常处理栈以及设置
处理函数为\_\_asm\_pgfault\_handler。

说到这里我们需要整理一下思路,fork中父进程在syscall\_env\_alloc后还需要做的事情有:
\textbf{地址空间的遍历以及duppage}、\textbf{分配子进程的异常处理栈}、\textbf{设置子进程的处理函数}、
\textbf{设置子进程的运行状态}。最后再将子进程的envid返回就大功告成了。

\begin{exercise}
 填写 user/fork.c 中的fork 函数中关于“父进程”执行的部分
\end{exercise}

至此,我们的lab4实验已经基本完成了,接下来就一起来愉快地调试吧!

\section{实验正确结果}

本次测试分为两个文件,当基础系统调用与fork写完后,单独测试fork的文件是\textbf{user/fktest.c},测试时将

ENV\_CREATE(user\_fktest)加入init.c即可测试。

正确结果如下:

\begin{minted}[linenos]{c}

	main.c:	main is start ...

	init.c:	mips_init() is called

	Physical memory: 65536K available, base = 65536K, extended = 0K

	to memory 80401000 for struct page directory.

	to memory 80431000 for struct Pages.

	mips_vm_init:boot_pgdir is 80400000

	pmap.c:	 mips vm init success

	panic at init.c:31: ^^^^^^^^^^^^^^^^^^^^^^^^^^^^^^^^^^^^^

	pageout:	@@@___0x7f3fe000___@@@  ins a page

	this is father: a:1

	this is father: a:1

	this is father: a:1

	this is father: a:1

	this is father: a:1

	this is father: a:1

	this is father: a:1

	this is father: a:1

	this is father: a:1

	  child :a:2

		this is child :a:2

		this is child :a:2

				this is child2 :a:3

				this is child2 :a:3

				this is child2 :a:3

				this is child2 :a:3

	this is father: a:1

	this is father: a:1

	this is father: a:1

	this is father: a:1

	this is father: a:1

		this is child :a:2

		this is child :a:2

		this is child :a:2
\end{minted}

另一个测试文件主要测试进程间通信,文件为\textbf{user/pingpong.c},测试方法同上。

正确结果如下:

\begin{minted}[linenos]{c}

main.c:	main is start ...

init.c:	mips_init() is called

Physical memory: 65536K available, base = 65536K, extended = 0K

to memory 80401000 for struct page directory.

to memory 80431000 for struct Pages.

mips_vm_init:boot_pgdir is 80400000

pmap.c:	 mips vm init success

panic at init.c:31: ^^^^^^^^^^^^^^^^^^^^^^^^^^^^^^^^^^^^^

pageout:	@@@___0x7f3fe000___@@@  ins a page


@@@@@send 0 from 800 to 1001

1001 am waiting.....

800 am waiting.....

1001 got 0 from 800

@@@@@send 1 from 1001 to 800

1001 am waiting.....

800 got 1 from 1001



@@@@@send 2 from 800 to 1001

800 am waiting.....

1001 got 2 from 800



@@@@@send 3 from 1001 to 800

1001 am wa800 got 3 from 1001



@@@@@send 4 from 800 to 1001

iting.....

800 am waiting.....

1001 got 4 from 800



@@@@@send 5 from 1001 to 800

1001 am waiting.....

800 got 5 from 1001



@@@@@send 6 from 800 to 1001

800 am waiting.....

1001 got 6 from 800



@@@@@send 7 from 1001 to 800

1001 am waiting.....

800 got 7 from 1001



@@@@@send 8 from 800 to 1001

800 am waiting.....

1001 got 8 from 800



@@@@@send 9 from 1001 to 800

1001 am waiting.....

800 got 9 from 1001



@@@@@send 10 from 800 to 1001

[00000800] destroying 00000800

[00000800] free env 00000800

i am killed ...

1001 got 10 from 800

[00001001] destroying 00001001

[00001001] free env 00001001

i am killed ...

\end{minted}

\section{实验思考}

\begin{itemize}
	\item \hyperref[think-syscall]{\textbf{\textcolor{baseB}{思考-系统调用的实现}}}
	\item \hyperref[think-father-son]{\textbf{\textcolor{baseB}{思考-不同的进程代码执行}}}
	\item \hyperref[think-fork的调用]{\textbf{\textcolor{baseB}{思考-fork的返回结果}}}
	\item \hyperref[think:遍历页]{\textbf{\textcolor{baseB}{思考-用户空间的保护}}}
	\item \hyperref[think:vpt的使用]{\textbf{\textcolor{baseB}{思考-vpt的使用}}}
	\item \hyperref[think:pgfault-kernel]{\textbf{\textcolor{baseB}{思考-缺页中断-内核处理}}}
	\item \hyperref[think:pgfault-user-1]{\textbf{\textcolor{baseB}{思考-缺页中断-用户处理-1}}}
	\item \hyperref[think:pgfault-user-2]{\textbf{\textcolor{baseB}{思考-缺页中断-用户处理-2}}}
	
	
\end{itemize}

