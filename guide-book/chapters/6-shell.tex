\chapter{管道与Shell}
%目标:user/pipe.c 与 user/sh.c
%是否新增?
%是否需要自行编写测试文件?/个人觉得这里应该要求自己编写测试文件进行测试(说明白原理即可)
\section{实验目的}
\begin{enumerate}
	\item 掌握管道的原理与底层细节
	\item 实现管道的读写
	\item 复述管道竞争情景
	\item 实现基本shell
	\item 实现shell中涉及管道的部分
\end{enumerate}

\section{管道}

在lab4中,我们已经学习过一种进程间通信(IPC,Inter-Process Communication)的方式——共享内存。
而今天我们要学的管道,其实也是进程间通信的一种方式。

\subsection{初窥管道}

通俗来讲,管道就像家里的自来水管:一端用于注入水,一端用于放出水,且水只能在一个方向上流动,而不能双向流动,所以说管道是典型的单向通信。管道又叫做匿名管道,只能用在具有公共祖先的进程之间使用,通常使用在父子进程之间通信。
 
在Unix中,管道由pipe函数创建,函数原型如下:

\begin{minted}[linenos]{c}
	#include<unistd.h>
	
	int  pipe(int fd[2]); 成功返回0,否则返回-1;
	
	参数fd返回两个文件描述符,fd[0]对应读端,fd[1]对应写端。
\end{minted}

为了更好地理解管道实现的原理,同样,我们先来做实验亲自体会一下\footnote{实验代码参考 http://pubs.opengroup.org/onlinepubs/9699919799/functions/pipe.html}

\begin{codeBoxWithCaption}{管道示例\label{code:test_pipe.c}}
	\inputminted[linenos]{c}{codes/test_pipe.c}
\end{codeBoxWithCaption}

示例代码实现了从父进程向子进程发送消息"Hello,world",并且在子进程中打印到屏幕上。它演示了管道在父子进程之间通信的基本用法:在pipe函数之后,调用fork来产生一个子进程,之后在父子进程中执行不同的操作。在示例代码中,父进程操作写端,而子进程操作读端。同时,示例代码也为我们演示了使用pipe系统调用的习惯:fork之后,进程在开始读或写管道之前都会关掉不会用到的管道端。

从本质上说,管道是一种只在内存中的文件。在UNIX中使用pipe系统调用时,进程中会打开两个新的文件描述符:一个只读端和一个只写端,而这两个文件描述符都映射到了同一片内存区域。但这样建立的管道的两端都在同一进程中,而且构建出的管道两端是两个匿名的文件描述符,这就让其他进程无法连接该管道。在fork的配合下,才能在父子进程间建立起进程间通信管道,这也是匿名管道只能在具有亲缘关系的进程间通信的原因。

\begin{thinking}\label{think-father-reader}
	示例代码中,父进程操作管道的写端,子进程操作管道的读端。如果现在想让父进程作为“读者”,代码应当如何修改?
\end{thinking}

\subsection{管道的测试}

我们下面就来填充函数实现匿名管道的功能。思考刚才的代码样例,要实现匿名管道,至少需要有两个功能:管道读取、管道写入。

要想实现管道,首先我们来看看本次实验我们将如何测试。lab6关于管道的测试有两个,分别是\mintinline{console}|user/testpipe.c|与\mintinline{console}|user/testpiperace.c|。

首先我们来观察testpipe的内容

\begin{codeBoxWithCaption}{testpipe测试\label{code:lab_test_pipe.c}}
	\inputminted[linenos]{c}{codes/lab_test_pipe.c}
\end{codeBoxWithCaption}

实际上可以看出,测试文件使用pipe的流程和示例代码是一致的。先使用函数 \mintinline{c}|pipe(int p[2]) |创建了管道,读端的文件控制块编号\footnote{文件控制块编号是int型,user/fd.c 中 num2fd 函数可通过它定位文件控制块的地址。}为p[0],写端的文件控制块编号为p[1]。之后使用fork()创建子进程,\textbf{注意这时父子进程使用p[0]和p[1]访问到的内存区域一致}。之后子进程关闭了p[1],从p[0]读;父进程关闭了p[0],从p[1]写入管道。

lab4的实验中,我们的fork实现是完全遵循Copy-On-Write原则的,即对于所有用户态的地址空间都进行了PTE\_COW的设置。
但实际上写时复制并不完全适用,至少在我们当前情景下是不允许写时拷贝。为什么呢?我们来看看pipe函数中的关键部分就能知晓答案:

\begin{minted}[linenos]{c}
int
pipe(int pfd[2])
{
	int r, va;
	struct Fd *fd0, *fd1;
	
	if ((r = fd_alloc(&fd0)) < 0
	||  (r = syscall_mem_alloc(0, (u_int)fd0, PTE_V|PTE_R|PTE_LIBRARY)) < 0)
	goto err;
	
	if ((r = fd_alloc(&fd1)) < 0
	||  (r = syscall_mem_alloc(0, (u_int)fd1, PTE_V|PTE_R|PTE_LIBRARY)) < 0)
	goto err1;
	
	va = fd2data(fd0);
	if ((r = syscall_mem_alloc(0, va, PTE_V|PTE_R|PTE_LIBRARY)) < 0)
	goto err2;
	if ((r = syscall_mem_map(0, va, 0, fd2data(fd1), PTE_V|PTE_R|PTE_LIBRARY)) < 0)
	goto err3;
	
	...
}
\end{minted}

在pipe中,首先分配两个文件描述符并为其分配空间,然后将一个管道作为这两个文件描述符数据区的第一页数据,从而使得这两个文件描述符能够共享一个管道的数据缓冲区。

\begin{exercise}
	仔细观察pipe中新出现的权限位\mintinline{c}|PTE_LIBRARY|,根据上述提示修改fork系统调用,使得\textbf{管道缓冲区是父子进程共享的},不设置为写时复制的模式。
\end{exercise}

下面我们使用一张图来表示父子进程与管道的数据缓冲区的关系:

\begin{figure}[htbp]
	\centering
	\includegraphics[width=15cm]{6-pipe-after-fork}
	\caption{父子进程与管道缓冲区}\label{fig:6-pipe-after-fork} 
\end{figure}

实际上,在父子进程中各自close掉不再使用的端口后,父子进程与管道缓冲区的关系如下图:

\begin{figure}[htbp]
	\centering
	\includegraphics[width=15cm]{6-pipe-after-close}
	\caption{关闭不使用的端口后}\label{fig:6-pipe-after-close} 
\end{figure}

下面我们来讲一下\mintinline{c}|struct Pipe|,并开始着手填写操作管道端的函数。

\subsection{管道的读写}

我们可以在 user/pipe.c 中轻松地找到Pipe结构体的定义,它的定义如下:

\begin{minted}[linenos]{c}
	struct Pipe {
		u_int p_rpos;		    // read position
		u_int p_wpos;		    // write position
		u_char p_buf[BY2PIPE];	// data buffer
	};
\end{minted}

在Pipe结构体中,p\_rpos给出了下一个将要从管道读的数据的位置,而p\_wpos给出了下一个将要向管道写的数据的位置。只有读者可以更新p\_rpos,同样,只有写者可以更新p\_wpos,读者和写者通过这两个变量的值进行协调读写。一个管道有BY2PIPE(32Byte)大小的缓冲区。

这个只有BY2PIPE大小的缓冲区发挥的作用类似于环形缓冲区,所以下一个要读或写的位置i实际上是i\%BY2PIPE。

读者在从管道读取数据时,要将p\_buf[p\_rpos\%BY2PIPE]的数据拷贝走,然后读指针自增1。但是需要注意的是,管道的缓冲区此时可能还没有被写入数据。所以如果管道数据为空,即当 p\_rpos >= p\_wpos时,应该进程切换到写者运行。

类似于读者,写者在向管道写入数据时,也是将数据存入p\_buf[p\_wpos\%BY2PIPE],然后写指针自增1。 需要注意管道的缓冲区可能出现满溢的情况,所以写者必须得在 p\_wpos - p\_rpos < BY2PIPE时方可运行,否则要一直挂起。

上面这些还不足以使得读者写者一定能顺利完成管道操作。假设这样的情景:管道写端已经全部关闭,读者读到缓冲区有效数据的末尾,此时有 p\_rpos = p\_wpos。按照上面的做法,我们这里应当切换到写者运行。但写者进程已经结束,进程切换就造成了死循环,这时候读者进程如何知道应当退出了呢?

为了解决上面提出的问题,我们必须得知道管道的另一端是否已经关闭。不论是在读者还是在写者中,我们都需要对另一端的状态进行判断:当出现缓冲区空或满的情况时,要根据另一端是否关闭来判断是否要返回。如果另一端已经关闭,进程返回0即可;如果没有关闭,则切换到其他进程运行。

\begin{note}
Unix : If all file descriptors referring to the write end of a pipe have been closed, then an attempt to read(2) from the pipe will see end-of-file (read(2) will return 0) link : http://linux.die.net/man/7/pipe
\end{note}

那么我们该如何知晓管道的另一端是否已经关闭了呢?这时就要用到我们的\mintinline{c}|static int _pipeisclosed(struct Fd *fd, struct Pipe *p)|函数。而这个函数的核心,就是下面我们要讲的恒成立等式了。

在之前的图\ref{fig:6-pipe-after-close}中我们没有明确画出文件描述符所占的页,但实际上,对于每一个匿名管道而言,我们分配了三页空间:一页是读数据的文件描述符rfd,一页是写数据的文件描述符wfd,剩下一页是被两个文件描述符共享的管道数据缓冲区。既然管道数据缓冲区h是被两个文件描述符所共享的,我们很直观地就能得到一个结论:如果有1个读者,1个写者,那么管道将被引用2次,就如同上图所示。pageref函数能得到页的引用次数,所以实际上有下面这个等式成立:

pageref(rfd) + pageref(wfd) = pageref(pipe)\label{variant}

\begin{note}
内核会对pages数组成员维护一个页引用变量 pp\_ref 来记录指向该物理页的虚页数量。pageref的实现实际上就是查询虚页 P 对应的实际物理页,然后返回其 pp\_ref 变量的值。
\end{note}

这个等式对我们而言有什么用呢?假设我们现在在运行读者进程,而进行管道写入的进程都已经结束了,那么此时就应该有:\mintinline{c}|pageref(wfd) = 0|。所以就有\mintinline{c}|pageref(rfd) = pageref(pipe)|。所以我们只要判断这个等式是否成立就可以得知写端是否关闭,对写者来说同理。

\begin{exercise}
	根据上述提示与代码中的注释,填写 user/pipe.c 中的 piperead、pipewrite、\_pipeisclosed 函数并通过 testpipe 的测试。
\end{exercise}

\begin{note}
注意在本次实验中由于文件系统服务所在进程已经默认为1号进程(起始进程为0号进程),在测试时想启用文件系统需要注意ENV\_CREATE(fs\_serv) 在 init.c 中的位置。
\end{note}

\subsection{管道的竞争}

我们的小操作系统采用的是时间片轮转调度的进程调度算法,这点你应该在lab3中就深有体会了。这种抢占式的进程管理就意味着,用户进程随时有可能会被打断。

当然,如果进程间是孤立的,随时打断也没有关系。但当多个进程共享同一个变量时,执行同一段代码,不同的进程执行顺序有可能产生完全不同的结果,造成运行结果的不确定性。而进程通信需要共享(不论是管道还是共享内存),所以我们要对进程中共享变量的读写操作有足够高的警惕。

实际上,因为管道本身的共享性质,所以在管道中有一系列的竞争情况。在当前这种不加锁控制的情况下,我们无法保证\mintinline{c}|_pipeisclosed|用于管道另一端关闭的判断一定返回正确的结果。

我们重新看之前写的\mintinline{c}|_pipeisclosed|函数。在这个函数中我们对\mintinline{c}|pageref(fd structure)|与 
\mintinline{c}|pageref(pipe structure)|进行了等价关系的判断。假如不考虑进程竞争,不论是在读者还是写者进程中,我们会认为:

\begin{itemize}
	\item 对fd和对pipe的pp\_ref 的\textbf{写入}是同步的。
	\item 对fd和对pipe的pp\_ref 的\textbf{读取}是同步的。 
\end{itemize}

但现在我们处于进程竞争、执行顺序不定的情景下,上述两种情况现在都会出现不同步的现象。想想看,如果在下面这种场景下,我们前面提到的等式\ref{variant}还是恒成立的吗:

\label{code:example-pipe}
\begin{minted}[linenos]{c}
	pipe(p);
	if(fork() == 0 ){
		close(p[1]);
		read(p[0],buf,sizeof buf);
	}else{
		close(p[0]);
		write(p[1],"Hello",5);
	}
\end{minted}

\begin{itemize}
	\item  fork结束后,子进程先执行。时钟中断产生在close(p[1])与read之间,父进程开始执行。
	\item 父进程在close(p[0])中,p[0]已经解除了对pipe的映射(unmap),还没有来得及解除对p[0]的映射,时钟中断产生,子进程接着执行。
	\item 注意此时各个页的引用情况: pageref(p[0]) = 2(因为父进程还没有解除对p[0]的映射),而pageref(p[1]) = 1(因为子进程已经关闭了p[1])。但注意,此时pipe的pageref是2,子进程中p[0]引用了pipe,同时父进程中p[0]刚解除对pipe的映射,所以在父进程中也只有p[1]引用了pipe。
	\item 子进程执行read,read中首先判断写者是否关闭。比较pageref(pipe)与pageref(p[0])之后发现它们都是2,说明写端已经关闭,于是子进程退出。
\end{itemize}

\begin{thinking}\label{think-dup}
	上面这种不同步修改pp\_ref而导致的进程竞争问题在 user/fd.c 中的dup函数中也存在。请结合代码模仿上述情景,分析一下我们的dup函数中为什么会出现预想之外的情况?
\end{thinking}

%问题的答案也简单,实际上是在dup函数的两次map之间,正确的顺序应该是先unmap fd,再unmap pipe。

那看到这里你有可能会问:在close中,既然问题出现在两次unmap之间,那么我们为什么不能使两次unmap统一起来是一个原子操作呢?要注意,在我们的小操作系统中,只有syscall\_开头的\textbf{系统调用函数}是原子操作,其他所有包括fork这些函数%\footnote{这里我们提到fork是“库函数”,好像与Linux里对fork是系统调用的措辞是矛盾的。但笔者认为,在我们的小操作系统中fork不算是系统调用。如果你持反对意见可以在报告中提出来。}
都是可能会被打断的。一次系统调用只能unmap一页,所以我们是不能保持两次unmap为一个原子操作的。那是不是一定要两次unmap是原子操作才能使得\mintinline{c}|_pipeisclosed|一定返回正确结果呢?

\begin{thinking}\label{think-automatic}
	阅读上述材料并思考:为什么系统调用一定是原子操作呢?如果你觉得不是所有的系统调用都是原子操作,请给出反例。希望能结合相关代码进行分析。
\end{thinking}

答案当然是否定的,\mintinline{c}|_pipeisclosed|函数返回正确结果的条件其实只是:

\begin{itemize}
	\item 写端关闭 当且仅当 pageref(p[0]) == pageref(pipe);
	\item 读端关闭 当且仅当 pageref(p[1]) == pageref(pipe);
\end{itemize}

比如说第一个条件,写端关闭时,当然有pageref(p[0]) == pageref(pipe)。所以我们要解决的实际上是 \textbf{当 pageref(p[0]) == pageref(pipe) 时,写端关闭}。正面如果不好解决 问题,我们可以考虑从其逆否命题着手,即要满足:{ 当写端没有关闭的时候, pageref(p[0]) $\neq$ pageref(pipe)}。

我们考虑之前那个预想之外的情景,它出现的最关键原因在于:pipe的引用次数总比fd要高。当管道的close进行到一半时,\textbf{若先解除pipe的映射,再解除fd的映射},就会使得pipe的引用次数的-1先于fd。这就导致在两个unmap的间隙,会出现pageref(pipe) == pageref(fd)的情况。那么若调换fd和pipe在close中的unmap顺序,能否解决这个问题呢?

\begin{thinking}\label{think-race}
	仔细阅读上面这段话,并思考下列问题
	\begin{itemize}
		\item 
		按照上述说法控制\textbf{pipeclose}中fd和pipe unmap的顺序,是否可以解决上述场景的进程竞争问题?给出你的分析过程。
		\item
		我们只分析了close时的情形,那么对于dup中出现的情况又该如何解决?请模仿上述材料写写你的理解。
	\end{itemize}
\end{thinking}

根据上面的描述我们其实已经能够得出一个结论:控制fd与pipe的map/unmap的顺序可以解决上述情景中出现的进程竞争问题。

那么下面根据你所思考的内容进行实践吧:

\begin{exercise}
	修改 user/pipe.c 中的 pipeclose 与 user/fd.c 中的 dup 函数 以避免上述情景中的进程竞争情况。
\end{exercise}

我们通过控制修改pp\_ref的前后顺序避免了“写数据”导致的错觉,但是我们还得解决第二个问题:
读取pp\_ref的同步问题。

同样是上面的代码\ref{code:example-pipe},我们思考下面的情景:

\begin{itemize}
	\item fork结束后,子进程先执行。执行完close(p[1])后,执行read,要从p[0]读取数据。但由于此时管道数据缓冲区为空,所以read函数要判断父进程中的写端是否关闭,进入到\_pipeisclosed函数,pageref(fd)值为2(父进程和子进程都打开了p[0]),时钟中断产生。
	\item 内核切换到父进程执行,父进程close(p[0]),之后向管道缓冲区写数据。要写的数据较多,写到一半时钟中断产生,内核切换到子进程运行。
	\item 子进程继续运行,获取到pageref(pipe)值为2(父进程打开了p[1],子进程打开了p[0]),引用值相等,于是认为父进程的写端已经关闭,子进程退出。
\end{itemize}

上述现象出现的根源在哪里呢?fd是一个父子进程共享的变量,但子进程中的pageref(fd)没有随父进程对fd的修改而同步,这就造成了子进程读到的pageref(fd)成为了“脏数据”。为了保证读的同步性,子进程应当重新读取pageref(fd)和pageref(pipe),并且要在\textbf{确认两次读取之间进程没有切换}后,才能返回正确的结果。为了实现这一点,我们要使用到之前一直都没用到的变量:env\_runs。

env\_runs记录了一个进程env\_run的次数,这样我们就可以根据某个操作do()前后进程env\_runs值是否相等,来判断在do()中进程是否发生了切换。

\begin{exercise}
	根据上面的表述,修改\mintinline{c}|_pipeisclosed|函数,使得它满足“同步读”的要求。注意env\_runs变量是需要维护的。
\end{exercise}

\section{shell}

shell本质上也是一个用户进程。它解释shell命令的工作是通过创建并运行子进程来完成的:对于每个shell命令,都有一个对应的可执行文件来完成该命令所要完成的工作,shell需要根据所得到的命令来创建执行相应可执行文件的子进程,从而完成命令的解释工作并得到结果。

实现shell的一个关键函数是spawn,其作用是接受可执行文件路径和传给其的参数,创建一个新的进程使得可执行文件的相关段被映射到新进程的内存中(可执行文件的相关段位置需要借助readelf命令的帮助,该命令同交叉编译器位于同一目录下。可执行文件的后缀为 .b),并设置好栈,以及一些已初始化的数据,比如管道和重定向的文件描述符。其作用类似load\_icode和fork,但实现方式不一样。相信经过之前的训练,同学们应该可以较为容易的填出spawn函数。

\begin{exercise}
  根据上面的表述以及函数内的注释,补充完成 user/spawn.c 中的 \mintinline{c}|int spawn(char *prog, char **argv)| 函数。\\
  完成后可以初步看见shell的现象。
\end{exercise}

\begin{thinking}\label{think-spawn}
  请解释\mintinline{c}|spawn|函数中注释标记为Share memory一段的作用,并说明为什么该段代码是正确的。你可以尝试对该段代码进行改动以探究其对运行结果的影响。
\end{thinking}

接下来,我们需要在shell进程里实现对管道和重定向的解释功能。解释shell命令时:

\begin{enumerate}
	\item 如果碰到重定向符号‘<’或者’>’,则读下一个单词,打开这个单词所代表的文件,然后将其复制给标准输入或者标准输出。
	\item 如果碰到管道符号’|’,则首先需要建立管道pipe,然后fork。
	\begin{itemize}
		\item 对于父进程,需要将管道的写者复制给标准输出,然后关闭父进程的读者和写者,运行‘|’左边的命令,获得输出,然后等待子进程运行。
		\item 对于子进程,将管道的读者复制给标准输入,从管道中读取数据,然后关闭子进程的读者和写者,继续读下一个单词。
	\end{itemize}
\end{enumerate}

\begin{exercise}
	根据以上描述,补充完成 user/sh.c 中的 \mintinline{c}|void runcmd(char *s)|。
\end{exercise}

\section{实验正确结果}

\subsection{管道测试}
管道测试有两个文件,分别是 user/testpipe.c 和 user/testpiperace.c ,以合适的次序建好进程后,在testpipe的测试中若出现两次\textbf{pipe tests passed}即说明测试通过。在testpiperace的测试中应当出现{race didn't happen}是正确的。

\subsection{shell测试}
在 init/init.c 中按照如下顺序依次启动 shell 和 文件服务:

\begin{minted}[linenos]{c}
	ENV_CREATE(user_icode);
	ENV_CREATE(fs_serv);
\end{minted}

如果正常会看到如下现象:

\begin{minted}[linenos]{c}
	:::::::::::::::::::::::::::::::::::::::::::::::::::::::::::::
	
	::                                                         ::
	
	::              Super Shell  V0.0.0_1                      ::
	
	::                                                         ::
	
	:::::::::::::::::::::::::::::::::::::::::::::::::::::::::::::
	\$
\end{minted}

使用不同的命令会有不同的效果:
\begin{itemize}
	\item 输入ls.b,会显示一些文件和文件夹;
	\item 输入cat.b,会有回显现象出现;
	\item 输入ls.b | cat.b,和 ls.b 的现象应当一致;
\end{itemize}

\section{实验思考}

\begin{itemize}
	\item \hyperref[think-father-reader]{\textbf{\textcolor{baseB}{思考-父进程为读者}}}
	\item \hyperref[think-dup]{\textbf{\textcolor{baseB}{思考-dup中的进程竞争}}}
	\item \hyperref[think-automatic]{\textbf{\textcolor{baseB}{思考-原子操作}}}
	\item \hyperref[think-race]{\textbf{\textcolor{baseB}{思考-解决进程竞争}}}
	\item \hyperref[think-spawn]{\textbf{\textcolor{baseB}{思考-spawn中的内存共享}}}
\end{itemize}





%接下来的书写思路:首先是根据testpipe.c里用到的东西来进行,然后展示pipe函数(从里面挖几个点用于),之后就是根据read和write里的面向对象形式的读写设备方式,
%开始让他们填写管道的读函数与写函数。读写直接根据注释填写即可,值得重点介绍的地方在于锁的那个地方。
